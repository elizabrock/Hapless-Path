\documentclass{beamer}
\mode<presentation>{} 
\title{Project Proposal}
\author{Eliza Brock}
\date{\today}
%\includeonlyframes{current}
\AtBeginSection[]{} % for optional outline or other recurrent slide
\mode<presentation>
{ \usetheme{Malmoe} }
\begin{document}

\begin{frame}
\titlepage
\end{frame}

\section{Introduction}
\begin{frame}
\frametitle{Background}
This project comes from the list of project ideas from the course syllabus:
\begin{quote}
XPath Evaluator: This will be a pedagogical tool aimed at helping students
understand the use of XPath. The objective of the project will be to develop a stand
tool that can be integrated with a web browser to evaluate XPath Expressions.
\end{quote}
\end{frame}

%\begin{frame}
%\frametitle{Goal}
%\begin{description} \item[Goal]... to teach! \end{description}
%\end{frame}

\begin{frame}
\frametitle{Introduction to the Project}
\begin{itemize}
\item Pedagogical tool
\pause
\item Dynamic visualization of XPath queries
%\item The primary objective of the XPath Evaluator is to serve as a pedagogical tool to illustrate the basic principles of querying XML using XPath.  
%\item It will generate a dynamic visual representation of XML data and XPath expressions in order to represent the effect of each individual XPath subquery on the transaction as a whole.
\pause
\item It will be badass.
\end{itemize}
\end{frame}

\section{Approach}
\begin{frame}
\frametitle{Approach}
%It's drawing on the board time!!!
%\begin{table}[!htpb]
%\caption{Proposed Workflow}
%\label{browser-interaction}
%\centering
\begin{tabular*}{\textwidth}{p{.45\textwidth} | p{.45\textwidth}}
\textbf{Client}					&	\textbf{Server}\\
\hline
\hline
User selects XML file to use	&	\\
\hline
								&	Graphical representation of the XML is generated \\
\hline
User inputs XPath, which is sent to the server on a per-token basis & \\
\hline
								& Identifiers of elements selected by individual
 subexpression are returned to the client \\
\hline The browser highlights the elements and corresponding expressions & \\
\end{tabular*}
%\end{table}
\end{frame}



\begin{frame}
\frametitle{Exclusions}
What is not included in my project? Well...
\begin{itemize}
\item The parts of XPath that I think are 
\begin{itemize}
\item (a) not that useful and
\item (b) a pain to implement.
\end{itemize}
\item Anything else that I think a proper academic wouldn't do. \\ For example:
\begin{itemize}
\item Cross browser support
\item Q.A.
\end{itemize}
\end{itemize}

\end{frame}


\section{Resources}
\begin{frame}
\frametitle{Papers/Resources}
I plan to read a lot of papers...\\
They wouldn't all fit on here, so here are my favorites: 
\nocite{chen:indexing}
\nocite{benedikt:structural}
\nocite{scardina:building}
%\renewcommand\refname{\section{Papers/Resources I plan to review }}
%\bibliographystyle{acm}
\bibliographystyle{plain} 
\bibliography{uno}
\end{frame}


\section{Timeline}
\begin{frame}
\frametitle{Timeline}
\begin{tabular}{l|l}
Week 	& Goals [Tasks] 							\\	
\hline
\hline
4.5th	& Templates of reports 			\\
 		& Finding suitable XML with DTDs 		\\
\hline
5th		& Parsing XML into datastructure \\
		& Parsing XPath by token 			\\
\hline
6th		& Visual display of XML elements 	\\
\hline
7th		& Connect XPath \& XML 			\\
\hline
8th		& \emph{Margin of Error}							\\
\hline
9th		& Final features					\\
\hline
10th	& \emph{Margin of Error}							\\
\hline
\end{tabular}
\end{frame}


\end{document}


