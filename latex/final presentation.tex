\documentclass{beamer}
\mode<presentation>{} 
\title{Hapless Path XPath Evaluator}
\author{Eliza Brock}
\date{\today}
%\includeonlyframes{current}
\AtBeginSection[]{} % for optional outline or other recurrent slide
\mode<presentation>
{ \usetheme{Malmoe} }
\begin{document}

\begin{frame}
\titlepage
\end{frame}

\section{Introduction}
\begin{frame}
\frametitle{Premise}
Pedagogical tool to illustrate the basic principles of querying XML using XPath. 
\begin{itemize}
\pause
\item Browser based
\pause
\item Dynamic visualization based on sub-queries
\pause
\item Written in Haskell
\end{itemize}
\end{frame}

\section{Approach}
\begin{frame}
\frametitle{Tools}
\begin{description}
\item[HAppS] ``Haskell Application Server'' Features auto recompilation, optional state storage, and is designed for the "pattern of developing static web pages and using AJAX to populate them with dynamic content"
\pause
\item[Parsec] ``industrial strength, monadic parser combinator library for Haskell. '' \cite{parsec} i.e. entirely bad-ass library to facilitate writing parsers in Haskell.
\pause
\item[AJAX] XMLHttpRequest. Magic.
\end{description}
\end{frame}

\begin{frame}
\frametitle{Workflow}
\begin{tabular*}{\textwidth}{p{.45\textwidth} | p{.45\textwidth}}
\textbf{Client}					&	\textbf{Server}\\
\hline
\hline
User selects XML file to use	&	\\
\hline
								&	Graphical representation of the XML is generated \\
\hline
User inputs XPath, which is sent to the server on a per-token basis & \\
\hline
								& Identifiers of elements selected by individual
 subexpression are returned to the client \\
\hline The browser highlights the \\
elements and corresponding expressions & \\
\end{tabular*}
%\end{table}
\end{frame}

\begin{frame}
\frametitle{Approach}

\begin{itemize}
\item 3 pass approach
	\begin{enumerate}
	\item Parse XML
	\item Parse XPath
	\item Evaluate them together
	\end{enumerate}
\pause
\item Not as bad as it sounds
	\begin{itemize}
	\item Haskell is lazy
	\item Haskell remembers things
	\end{itemize}
\end{itemize}
\end{frame}

\section{Research}
\begin{frame}
\frametitle{Papers/Resources}
% I read a lot of papers...\\
% It was really quite sad.\\
% The ones I actually referenced were:

\nocite{parsec}
\nocite{xmlspec}
\nocite{xpathspec}
\nocite{designandimplement}
\nocite{xpathtokenizer}
%\renewcommand\refname{\section{Papers/Resources I plan to review }}
%\bibliographystyle{acm}
\bibliographystyle{plain} 
\bibliography{uno}
\end{frame}

\section{Results}

\begin{frame}
\frametitle{Implemented Features}

\begin{itemize}
\item XML
	\begin{itemize}
	\item elements with children
	\item empty elements
	\item comments
	\item doctype declarations
	\item mixed content
	\item attributes
	\end{itemize}
\pause
\item XPath
	\begin{itemize}
	\item 	abbreviated location paths (except  //)
	\item function calls
	\item predicates
	\end{itemize}
\pause
\item Evaluation
\begin{itemize}
	\item location steps
	\item predicate tests
	\item a subset of the function calls that result in nodes or node-sets
	\end{itemize}
\end{itemize}
\end{frame}

\section{Demo}
\begin{frame}
\frametitle{Demonstration}

\end{frame}
\end{document}


